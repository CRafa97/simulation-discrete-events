%===================================================================================
% PREÁMBULO
%-----------------------------------------------------------------------------------
\documentclass[a4paper,10pt,twocolumn]{article}

%===================================================================================
% Paquetes
%-----------------------------------------------------------------------------------
\usepackage{amsmath}
\usepackage{amsfonts}
\usepackage{amssymb}
\usepackage{informe}
\usepackage[utf8]{inputenc}
\usepackage{listings}
\usepackage{algorithmic}
\usepackage[pdftex]{hyperref}
%-----------------------------------------------------------------------------------
% Configuración
%-----------------------------------------------------------------------------------
\hypersetup{colorlinks,%
	    citecolor=black,%
	    filecolor=black,%
	    linkcolor=black,%
	    urlcolor=blue}

%===================================================================================



%===================================================================================
% Presentacion
%-----------------------------------------------------------------------------------
% Título
%-----------------------------------------------------------------------------------
\title{Simulaci\'on basada en Eventos Discretos}

%-----------------------------------------------------------------------------------
% Autores
%-----------------------------------------------------------------------------------
\author{\\
\name Carlos Rafael Ortega Lezcano \email \href{mailto:c.ortega@estudiantes.matcom.uh.cu}{c.ortega@estudiantes.matcom.uh.cu}
	\\ \addr Grupo C411}

%-----------------------------------------------------------------------------------
% Tutores
%-----------------------------------------------------------------------------------
\tutors{\\}

%-----------------------------------------------------------------------------------
% Headings
%-----------------------------------------------------------------------------------
%\jcematcomheading{\the\year}{1-\pageref{end}}{Carlos Rafael}

%-----------------------------------------------------------------------------------
%\ShortHeadings{Simulacio\'n basada en Eventos Discretos}{Carlos Rafael}
%===================================================================================



%===================================================================================
% DOCUMENTO
%-----------------------------------------------------------------------------------
\begin{document}

%-----------------------------------------------------------------------------------
% NO BORRAR ESTA LINEA!
%-----------------------------------------------------------------------------------
\twocolumn[
%-----------------------------------------------------------------------------------

\maketitle

%===================================================================================
% Resumen y Abstract
%-----------------------------------------------------------------------------------
\selectlanguage{spanish} % Para producir el documento en Español

%-----------------------------------------------------------------------------------
% Resumen en Español
%-----------------------------------------------------------------------------------
%\begin{abstract}

%	El Resumen en Español debe constar de $100$ a $200$ palabras y presentar de forma
%	clara y concisa el contenido fundamental del artículo.

%\end{abstract}

%-----------------------------------------------------------------------------------
% English Abstract
%-----------------------------------------------------------------------------------
\vspace{0.5cm}

%\begin{enabstract}

%  The English Abstract must have have $100$ to $200$ words, and present in a clear
%  and concise form the essentials of the article content.

%\end{enabstract}

%-----------------------------------------------------------------------------------
% Palabras clave
%-----------------------------------------------------------------------------------
%\begin{keywords}
%	Separadas,
%	Por,
%	Comas.
%\end{keywords}

%-----------------------------------------------------------------------------------
% Temas
%-----------------------------------------------------------------------------------
%\begin{topics}
%	Tema, Subtema.
%\end{topics}


%-----------------------------------------------------------------------------------
% NO BORRAR ESTAS LINEAS!
%-----------------------------------------------------------------------------------
\vspace{0.8cm}
]
%-----------------------------------------------------------------------------------


%===================================================================================

%===================================================================================
% Introducción
%-----------------------------------------------------------------------------------
\section{Problema Asignado}\label{sec:intro}
%-----------------------------------------------------------------------------------

En el Aeropuerto de Barajas, se desea conocer cu\'anto tiempo se encuentran vac\'ias
las pistas de aterrizaje dedicadas a aviones de carga y que se consideran que una pista
est\'a ocupada cuando hay un avi\'on aterrizando, despegando o cuando se encuentra cargando 
o descargando mercanc\'ia o el abordaje o aterrizaje de cada pasajero. \\
Se cononce que el tiempo cada avi\'on que arriba al aeropuerto distribuye, mediante una funci\'on
de distribuci\'on exponencial con $\lambda = 20$ minutos. \\
Si un avi\'on arriba al aeropuerto y no exiten pistas vac\'ias, se mantiene esperando hasta que se 
vac\'ie una de ellas (en caso de que existan varios aviones en esta situaci\'on, pues se establece una suerte
de cola para su aterrizaje). \\
Se conoce adem\'as que el tiempo de carga y descarga de un avi\'on distribuye mediante una funci\'on de 
distribuci\'on exponencial con $\lambda = 30$ minutos. Se considera adem\'as que el tiempo de aterrizaje y despegue de
un avi\'on distribuye normal $N(10,5)$ y la probabilidad de que un avi\'on cargue y/o descargue en cada viaje corresponde a una distribuci\'on
uniforme. \\
Adem\'as de esto se conoce que los aviones tienen una probabilidad de tener una rotura de $0.1$. As\'i, cuando un avi\'on posee alguna rotura debe
ser reparado en un tiempo que distribuye exponencial con $\lambda = 15$ minutos. Las roturas se identifican justo antes del despegue del avi\'on.\\
Igualmente cada avi\'on, durante el tiempo que est\'a en la pista debe recargar combustible, mediante una distribuci\'on exponencial con $\lambda = 30$
minutos y comienza justamente cuando el avi\'on aterriza.\\
Se asume adem\'as qu los aviones pueden aterrizar en cada pista sin ninguna preferencia o requerimiento. \\
Se desea simular el comportamiento del aeropuerto por una semana para estimar el tiempo total en que se encuentran vac\'ia cada una de las pistas del
aeropuerto.

%===================================================================================

%===================================================================================
% Desarrollo
%-----------------------------------------------------------------------------------
\section{Principales Ideas}\label{sec:dev}
%-----------------------------------------------------------------------------------
  
El problema corresponde un acercamiento a un sistema compuesto por clientes (aviones)
los cuales son atendidos por distintos servidores que funcionan en paralelo (pistas)
pudiendo encontrarse clientes que deban esperar para ser atentidos en el sistema.
En nuestro caso la cantidad de servidores que funcionan en paralelo es $n = 5$. \\

Como consideraciones adicionales podemos observar que el tiempo de estancia en una pista
es determinado por la combinaci\'on de distintas variables, adem\'as para conocer el tiempo
total para cada pista es necesario conocer el instante en que un avi\'on entra en una pista
y el instante en que abandona esta.

%-----------------------------------------------------------------------------------
	\subsection{Variables de la Simulaci\'on}\label{sub:results}
%-----------------------------------------------------------------------------------
	Las variables empleadas para el proceso de simulaci\'on del problema son las 
	siguientes, los aviones se identifican por un valor $Id$ el cual se asigna a cada 
	avi\'on que arriva al aeropuerto de forma tal que $Id$ = $N_{A}$:

	\begin{description}
		\item \textbf{Variable de Tiempo} ($t$): Describe el tiempo transcurrido hasta el momento en la simulaci\'on
		\item \textbf{Variable Contadora} ($N_{A}$): Describe la cantidad de aviones que han arribado al aeropuerto hasta el instante $t$
		\item \textbf{Variables de Estado}:
		\begin{description}
			\item $SS$: Contiene la informaci\'on del avi\'on que se encuentra en la pista $i$, $i = 1,2,3,4,5$
			\item $Q$: Contiene los aviones que esperan para poder aterrizar en una pista del aeropuerto 	
		\end{description}
		\item \textbf{Variables de Salida}:
		\begin{description}
			\item $A$: Contiene para cada pista $i$, $i = 1,2,3,4,5$, los tiempos en que cada avi\'on aterriza en la pista
			\item $D$: Contiene para cada pista $i$, $i = 1,2,3,4,5$, los tiempos de salida para cada avi\'on que estuvo en la pista 	
		\end{description}
		\item \textbf{Lista de Eventos}:
		\begin{description}
			\item $t_{A}$: Representa el tiempo de arribo de un nuevo avi\'on al aeropuerto
			\item $t_{D}$: Almacena para cada pista $i$, el tiempo de partida del avi\'on que se encuentra en esta
		\end{description}
	\end{description}
	
	La variable $T$ representa el tiempo limite para la simulaci\'on y se pasa como entrada al programa.
	Se desea conocer el tiempo que cada pista permanece vac\'ia este tiempo es aquel que ocurre desde la salida de
	un avi\'on a la pista hasta que otro arriba a esta por lo tanto si $A$ son los tiempos de entrada a la pista
	para los aviones que estuvieron en esta y $D$ los tiempos de salida, el tiempo que una pista permanece vac\'ia
	puede determinarce como:

	\begin{align*}
		E = A[0] + \sum_{i=1}^{l}(A[i] - D[i-1])  
	\end{align*}

	Donde $l$ es el largo de $A$, $A[0]$ el tiempo que se espera hasta que arriba el primer avi\'on, y 
	la diferencia $A[i] - D[i-1]$ el tiempo que transcurre entre la salida del avi\'on $i-1$ hasta la llegada
	del $i$ a la pista.

%-----------------------------------------------------------------------------------
	\subsection{Variables Aleatorias presentes en la Simulaci\'on}\label{sub:lists}
%-----------------------------------------------------------------------------------
		
		El comportamiento del tiempo de llegada y de estancia de los aviones viene dado
		por un conjunto de diversas variables aleatorias.
		
		\begin{enumerate}
			\item Tiempo de arribo de un avi\'on al aeropuerto: $T_{0} \sim  Exp(20)$
			\item Tiempo de carga y descarga: $T_{cd} \sim Exp(30)$
			\item Tiempo de aterrizaje y despegue: $T_{ad} \sim N(10,5)$
			\item Probabilidad de Ruptura $P(R) = 0.1$, Tiempo de Reparaci\'on $T_{r} \sim Exp(15)$
			\item Tiempo de Recargar Combustible $T_{c} \sim Exp(30)$ 
		\end{enumerate}

		Para generar una variable aleatoria exponencial para poder describir el comportamiento de los diversos sucesos se emplea el m\'etodo de la inversa:
		
		\begin{algorithmic}[1]
			\STATE Generar un n\'umero aleatorio $U$
			\STATE Hacer $X = - \dfrac{1}{\lambda} log(U)$ 
			\STATE Retornar $X$
		\end{algorithmic}

		En el caso de la variable normal primero se genera $Y \sim N(0,1)$ y luego se aplica: $X = \mu + Y\sigma$. Finalmente para determinar si un avi\'on sufrir\'a una ruptura o no se emplea:
		
		\begin{align*}
			p\left(X = x_{j}\right) = p\left(\sum_{i=0}^{j-1} p_{i} \leq U < \sum_{i=0}^{j}p_{i} \right) = p_{j}
		\end{align*}

		Donde $U$ distribuye uniforme de 0 a 1. 


%-----------------------------------------------------------------------------------
	\subsection{Modelo}\label{sub:figures}
%-----------------------------------------------------------------------------------
		
		Dadas las variables de la simulaci\'on procedemos a realizar el proceso generando la llegada de los aviones y determinando cuanto tiempo pasaran en la pista.
		
		\newpage
		
		\textbf{Inicializaci\'on}
		\begin{enumerate}
			\item[] $t = N_{A} = 0$
			\item[] $Q = \{\}$
			\item[] $SS = \{i: 0\}$
			\item[] $A = \{i: \text{[ ]}\}, \;\; D = \{i: \text{[ ]}\}$
			\item[] Generar $T_{0} \sim Exp(20)$ y hacer $t_{A} = T_{0}$
			\item[] $t_{D} = \{i: \infty\}$
		\end{enumerate}

		\textbf{Caso 1} $ t_{A} = \min{( t_{A}, t_{D}[i] )}, i = \overline{1,5}$
		
		\begin{enumerate}
			\item[] $t = t_{A}$
			\item[] $N_{A} = N_{A} + 1$
			\item[] Generar $T_{t} \sim Exp(20)$ hacer $t_{A} = t + T_{t}$
			
			\textbf{IF} $\exists i \; | \; SS[i] = 0$, $i = \overline{1,5}$
				\begin{enumerate}
					\item[] $A[i] \leftarrow t$
					\item[] $SS[i] = N_{A}$
					\item[] Generar $Y = Depart()$ hacer $t_{D}[i] = t + Y$ 
				\end{enumerate}
			
			\textbf{IF} $SS[i] \neq 0$, $\forall i = \overline{1,5}$
			\begin{enumerate}
				\item[] $Q \leftarrow N_{A}$ 
			\end{enumerate}
		\end{enumerate}
	
		\textbf{Caso 2} $ t_{j} = \min{( t_{A}, t_{D}[i] )} \wedge t_{j} \neq t_{A}, i = \overline{1,5}$
		
		\begin{enumerate}
			\item[] $t_{j} = \min(t_{D})$
			\item[] $t = t_{j}$
			\item[] $D[j] \leftarrow t$
			\item[] $SS[i] = 0$
			
			\textbf{IF} $Q \neq \{\}$
				\begin{enumerate}
					\item[] $nxt = Q.pop()$
					\item[] $A[i] \leftarrow t$
					\item[] $SS[i] = nxt$
					\item[] Generar $Y = Depart()$ hacer $t_{D}[i] = t + Y$ 
				\end{enumerate} 
		\end{enumerate}
		
		\textbf{Caso 3} $ \min{( t_{A}, t, t_{D}[i] )} > T, i = \overline{1,5}$
		
		\begin{enumerate}
			\item[] Terminar la simulaci\'on 
		\end{enumerate}
		
		Para generar el tiempo que pasa un avi\'on en la pista se emplea $Depart$ que realiza las siguientes operaciones:\\
		
		\textbf{Depart}:
		\begin{enumerate}
			\item[] Generar $A \sim N(10,5)$ tiempo de aterrizaje
			\item[] Generar $D \sim N(10,5)$ tiempo de despegue
			\item[] Determinar si habr\'a rotura generando $U \sim U(0,1)$ y generar $F \sim Exp(15)$ hacer $F = F * U$
			\item[] Generar $R \sim Exp(30)$ tiempo de recarga de combustible
			\item[] Generar $C \sim Exp(30)$ tiempo de carga/descarga
			\item[] Hacer $M = \max{(R, C)}$
			\item[] Retornar $A + D + F + M$, tiempo de estancia en la pista
		\end{enumerate}

%-----------------------------------------------------------------------------------
\section{Consideraciones}
%-----------------------------------------------------------------------------------

%===================================================================================
	
	Una corrida de ejemplo de la simulaci\'on para una 
	se-mana (10080 minutos) muestra los siguientes resultados:
	\begin{center}
		\begin{tabular}[t]{|c|c|}
			\hline
			Pistas & Tiempo (minutos) \\ \hline
			Pista 1 & 935 \\ \hline
			Pista 2 & 1845 \\ \hline
			Pista 3 & 1533 \\ \hline
			Pista 4 & 2348 \\ \hline
			Pista 5 & 3000 \\ \hline
		\end{tabular}
	\end{center}

	Como los aviones no tienen preferencia por determinadas pistas adem\'as de que
	no existen requisitos para aterrizar en una pista, entonces se selecciona la primera
	pista que se encuentra vac\'ia debido a esto podemos observar como los tiempos de las
	pistas 4 y 5 son mayores, mientras la pista 1 es la que menos tiempo permanece libre.\\

	Para la selecci\'on de la pista se realiz\'o una corrida donde se selecciona de forma
	aleatoria la pista de aterrizaje resultando:
	\begin{center}
		\begin{tabular}[t]{|c|c|}
			\hline
			Pistas & Tiempo (minutos) \\ \hline
			Pista 1 & 2118 \\ \hline
			Pista 2 & 1775 \\ \hline
			Pista 3 & 2134 \\ \hline
			Pista 4 & 2508 \\ \hline
			Pista 5 & 2039 \\ \hline
		\end{tabular}
	\end{center}

	En este caso los tiempos son m\'as igualados ya que las primeras pistas no siempre
	reciben a un avi\'on que arriba o estaba en cola sino que la repartici\'on es aleatoria
	entre las pistas que est\'an libres.

%===================================================================================
% Conclusiones
%-----------------------------------------------------------------------------------
\section{Repositorio}\label{sec:conc}

	\href{https://github.com/CRafa97/simulation-discrete-events.git}{https://github.com/CRafa97/simulation-discrete-events.git}

%===================================================================================

%===================================================================================
% Bibliografía
%-----------------------------------------------------------------------------------
\begin{thebibliography}{99}
%-----------------------------------------------------------------------------------
	\bibitem{ross} Ross, Sheldon M. \emph{Simulation}.
		(5th~edition), 2013.
		Elsevier.
		
	\bibitem{rossprob} Ross, Sheldon M. \emph{A First Course in Probability}.
		(9th~edition), 2014.
		Pearson Education

%-----------------------------------------------------------------------------------
\end{thebibliography}

%-----------------------------------------------------------------------------------

\label{end}

\end{document}

%===================================================================================
